\chapter*{Resumen}

Desde la llegada del coronavirus Sars-Cov-2 o también conocido popularmente como Covid-19, el mundo ha cambiado de forma radical. Desde nuestra manera de relacionarnos hasta nuestro día a día. Algo que antes no parecía un problema como era la gestión de aforo en espacios cerrados, ahora es una gran cuestión para grandes y pequeños comercios desde gimnasios, cines, teatros o incluso para poder realizar una pequeña reunión con familiares y amigos. Anteriormente a la llegada de la pandemia, la gestión de aforo no era una medida sanitaria de urgencia y mucho menos un problema para los comercios en general como para sus usuarios, cierto es que ya existía un control de aforo para ciertas actividades de ocio como pueden ser  por ejemplo conciertos, o una presentación de teatro, pero no era una cuestión sanitaria como lo es hoy en día, era simplemente una criterio de si se habían vendido todas las localidades o había disponibilidad de asistir a la actividad.

El Sars-Cov-2 lleva ya 2 años entre nosotros. Desde su aparición en China no han dejado de existir diferentes variantes y mutaciones de este virus el cual ha producido que cada día existan mas restricciones a nivel global en todas las sociedades del mundo. Uno de los grandes problemas en la sociedad es el control de aforo en todas las actividades ya que no existe a día de hoy sistemas de recuento que nos ayuden a resolverlo de forma sencilla. Estos sistemas nos ayudaran para el Sars-Cov-2 y otro virus o pandemias que puedas existir en el mundo asi para paliar problemas que ocurran por el abarrotamiento de gente dentro de un lugar.

Una solución para poder medir el aforo de un espacio es mediante un Bot-chatbot, el cual nos dará a tiempo real la información necesaria de capacidad de cierto lugar que queramos consultar, quitándole el trabajo al personal del comercio de estar haciendo cuentas tanto positivas como negativas de las personas que estén asistiendo a dicho espacio.

Un Bot es un software informático destinado a la realización de tareas repetitivas con cierta inteligencia, como tareas cotidianas que hacen las personas en su día a día.
Uno de los bots más populares es un ChatBot, implementado en muchísimas empresas y sitios de Internet, se basa en resolver preguntas basándose en inteligencia artificial programada. Destaca por la manera de poder mantener una conversación con una persona humana, como también por la manera de poder ejecutar ciertas ordenes que le enseñemos previamente.
En el caso de la gestión de aforo en ciertos lugares, el ChatBot es una gran herramienta que puede ayudar a facilitar las operaciones de conteo de personas, teniendo asi un conteo a tiempo real de la capacidad de aforo de ciertos lugares como puede ser en un gimnasio. Los usuarios que quieran acceder al gimnasio a determinada hora podrán consultar a través del ChatBot la capacidad y tener asi una información importante con la cual podrán saber si el gimnasio tiene el aforo completo o aún tiene capacidad para acceder a él, ayudándole al usuario a tener una gestión de su tiempo de ocio en su día a día.

%%--------------
\newpage
%%--------------

\chapter*{Abstract}

<<Abstract of the Final Degree Project. Maximum length: 2 pages.>>


%%%%%%%%%%%%%%%%%%%%%%%%%%%%%%%%%%%%%%%%%%%%%%%%%%%%%%%%%%%
%% Final del resumen. 
%%%%%%%%%%%%%%%%%%%%%%%%%%%%%%%%%%%%%%%%%%%%%%%%%%%%%%%%%%%